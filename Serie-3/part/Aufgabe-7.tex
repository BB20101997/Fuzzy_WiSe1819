\section*{\underline{Aufgabe 7}}

\textit{Hinweis: Die rechtsbündigen Zahlen in Klammern verweisen auf die verwendeten Eigenschaften der t-Normen.}\\

\newtheorem{lem}{Lemma}
\begin{lem}
  Es gilt ${t}\left( 0,a \right) = 0$ für alle $a \in [0,1]$.
\end{lem}
\begin{proof}
  \begin{align*}
    &{t}\left( 0,1 \right) = 0 \tag{1}\\
    &0 \leq 0 \wedge a \leq 1 \Rightarrow {t}\left( 0,a \right) \leq {t}\left( 0,1 \right) \tag 2\\
    &{t}\left( 0,a \right) \leq 0 \Rightarrow {t}\left( 0,a \right) = 0
  \end{align*}
\end{proof}

\newtheorem{bew}{Beweis}
\begin{bew}[Aufgabe 7]

\end{bew}
\begin{proof}

Für die linke Ungleichung erhalten wir bei Seitentausch:

\begin{align*}
{t}\left( a,b \right)\geq\textit{dra}_t\left( a,b \right)=\left\{ 
\begin{array}{*{35}{l}}   
a & \textit{falls} & b=1, \\ 
b & \textit{falls} & a=1, \\   
0 & \textit{sonst} \\  
\end{array} 
\right.
\end{align*}

Wir zeigen die drei Beziehungen, die durch die Fallunterscheidung entstehen:

\begin{align*}
 \textit{Fall} \enspace b=1: \quad &{t}\left( a,1 \right) = a \geq a = \textit{dra}_t\left( a,1 \right) \tag 1\\
 \textit{Fall} \enspace a=1: \quad &{t}\left( 1,b \right) = {t}\left( b,1 \right) = b \geq b = \textit{dra}_t\left( b,1 \right) \tag{3, 1}\\
 \textit{Fall} \enspace a\neq 1, b\neq 1: \quad &\textit{Mit Lemma 1 folgt:}\\
  &{t}\left( a,b \right) \geq {t}\left( a,0 \right) = 0 = \textit{dra}_t\left( a,b \right) 
\end{align*}

Für die rechte Ungleichung erhalten wir:

\begin{align*}
&{t}\left( a,b \right) \leq {t}\left( a,1 \right) = a \tag{2, 1}\\
&{t}\left( a,b \right) \leq {t}\left( 1,b \right) = {t}\left( b,1 \right) = b \tag{2, 3, 1}\\
& {t}\left( a,b \right) \leq a \wedge {t}\left( a,b \right) \leq b \Rightarrow {t}\left( a,b \right) \leq \textit{min}\left\{ a,b \right\} = t_\textit{min}\left( a,b \right)
\end{align*}

\end{proof}