\section*{\underline{Aufgabe 9}}

\textit{Hinweis: Die rechtsbündigen Zahlen in Klammern verweisen auf die verwendeten Eigenschaften der t-Normen.}\\

\newtheorem{bew}{Beweis}
\begin{bew}[Aufgabe 9]

\end{bew}
\begin{proof}

Sei $t$ eine t-Norm, für die die Idempotenz gilt.
Seien weiterhin beliebige $a, b \in [0,1]$ mit $a \leq b$, dann gilt:
\begin{flalign*}
 &a = {t}\left( a,a \right) \leq \textit{t}\left( a,b \right) \leq \textit{t}\left( a,1 \right) = a \tag{3, 1}&\\
 &\Rightarrow \textit{t}\left( a,b \right) = a \\\\
 &\text{Durch Kommutativität gilt analog für } b \leq a \text{:} \tag{3}&\\ 
 &\textit{t}\left( a,b \right) = b
\end{flalign*}

Daraus folgt $\textit{t}\left( a,b \right) = \textit{min}\left( a,b \right)$ für alle $a, b \in [0,1]$.

\end{proof}