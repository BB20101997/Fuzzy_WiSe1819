\section*{\underline{Aufgabe 15}}

\subsection*{a)}

Die Clustering-Objekte sind die Zustände des Buchenwaldes zu den unterschiedlichen Untersuchungszeiten. Die Objekt-Merkmale sind die Flächendeckungswerte der 8 Vegetationsarten.

\subsection*{b)}

\subsubsection*{2 Cluster}

0.0767  \qquad  0.9233\\
0.0776  \qquad  0.9224\\
0.0776  \qquad  0.9224\\
0.0323  \qquad  0.9677\\
0.0147  \qquad  0.9853\\
0.0147  \qquad  0.9853\\
0.0211  \qquad  0.9789\\
0.3782  \qquad  0.6218\\
0.3839  \qquad  0.6161\\
0.9496  \qquad  0.0504\\
0.9496  \qquad  0.0504\\
0.9828  \qquad  0.0172\\
0.9827  \qquad  0.0173\\
0.8232  \qquad  0.1768\\
0.8011  \qquad  0.1989\\
0.0869  \qquad  0.9131\\
0.1907  \qquad  0.8093\\
0.1911  \qquad  0.8089\\
0.0581  \qquad  0.9419

\subsubsection*{3 Cluster}

0.0293  \qquad  0.0128  \qquad  0.9579\\
0.0297  \qquad  0.0131  \qquad  0.9572\\
0.0297  \qquad  0.0131  \qquad  0.9572\\
0.0767  \qquad  0.0205  \qquad  0.9028\\
0.0235  \qquad  0.0059  \qquad  0.9706\\
0.0235  \qquad  0.0059  \qquad  0.9706\\
0.0384  \qquad  0.0100  \qquad  0.9515\\
0.2044  \qquad  0.2500  \qquad  0.5456\\
0.2075  \qquad  0.2557  \qquad  0.5367\\
0.0192  \qquad  0.9534  \qquad  0.0274\\
0.0192  \qquad  0.9534  \qquad  0.0274\\
0.0037  \qquad  0.9916  \qquad  0.0047\\
0.0039  \qquad  0.9911  \qquad  0.0049\\
0.2035  \qquad  0.6288  \qquad  0.1677\\
0.2256  \qquad  0.5980  \qquad  0.1764\\
0.9526  \qquad  0.0077  \qquad  0.0396\\
0.9426  \qquad  0.0150  \qquad  0.0425\\
0.9425  \qquad  0.0150  \qquad  0.0424\\
0.8444  \qquad  0.0204  \qquad  0.1352

\subsubsection*{4 Cluster}

0.0135  \qquad  0.0091  \qquad  0.9554  \qquad  0.0220\\
0.0138  \qquad  0.0093  \qquad  0.9546  \qquad  0.0224\\
0.0138  \qquad  0.0093  \qquad  0.9546  \qquad  0.0224\\
0.0302  \qquad  0.0172  \qquad  0.8833  \qquad  0.0693\\
0.0086  \qquad  0.0048  \qquad  0.9665  \qquad  0.0202\\
0.0086  \qquad  0.0048  \qquad  0.9665  \qquad  0.0202\\
0.0152  \qquad  0.0085  \qquad  0.9412  \qquad  0.0350\\
0.3262  \qquad  0.1564  \qquad  0.3718  \qquad  0.1456\\
0.3250  \qquad  0.1603  \qquad  0.3667  \qquad  0.1480\\
0.0196  \qquad  0.9696  \qquad  0.0063  \qquad  0.0044\\
0.0196  \qquad  0.9696  \qquad  0.0063  \qquad  0.0044\\
0.0310  \qquad  0.9577  \qquad  0.0063  \qquad  0.0050\\
0.0335  \qquad  0.9543  \qquad  0.0067  \qquad  0.0054\\
0.9875  \qquad  0.0068  \qquad  0.0026  \qquad  0.0031\\
0.9727  \qquad  0.0142  \qquad  0.0058  \qquad  0.0073\\
0.0198  \qquad  0.0078  \qquad  0.0454  \qquad  0.9270\\
0.0263  \qquad  0.0101  \qquad  0.0325  \qquad  0.9312\\
0.0262  \qquad  0.0101  \qquad  0.0324  \qquad  0.9313\\
0.0463  \qquad  0.0189  \qquad  0.1416  \qquad  0.7931

\hfill



Es zeichnet sich ab das zu den Zeitraumen 14 und 15 , dem Zeitraum 10 bis 13, dem Zeitraum 1 bis 7 sowie dem Zeitraum 16 - 19 jeweils die Vegetationsarten in ähnlichen Verhältnissen auftreten.
Dies entspricht in etwa unserem Verständnis von den vier Jahreszeiten.

\subsection*{c)}

0.0004  \qquad  0.9990  \qquad  0.0001  \qquad  0.0005\\
0.0004  \qquad  0.9990  \qquad  0.0001  \qquad  0.0005\\
0.0004  \qquad  0.9990  \qquad  0.0001  \qquad  0.0005\\
0.0023  \qquad  0.9915  \qquad  0.0003  \qquad  0.0059\\
0.0002  \qquad  0.9993  \qquad  0.0000  \qquad  0.0005\\
0.0002  \qquad  0.9993  \qquad  0.0000  \qquad  0.0005\\
0.0006  \qquad  0.9978  \qquad  0.0001  \qquad  0.0015\\
0.7745  \qquad  0.1663  \qquad  0.0311  \qquad  0.0281\\
0.7714  \qquad  0.1656  \qquad  0.0333  \qquad  0.0296\\
0.0005  \qquad  0.0000  \qquad  0.9995  \qquad  0.0000\\
0.0005  \qquad  0.0000  \qquad  0.9995  \qquad  0.0000\\
0.0008  \qquad  0.0000  \qquad  0.9991  \qquad  0.0000\\
0.0010  \qquad  0.0000  \qquad  0.9990  \qquad  0.0000\\
0.9752  \qquad  0.0026  \qquad  0.0184  \qquad  0.0038\\
0.9662  \qquad  0.0039  \qquad  0.0236  \qquad  0.0063\\
0.0006  \qquad  0.0020  \qquad  0.0001  \qquad  0.9973\\
0.0011  \qquad  0.0013  \qquad  0.0001  \qquad  0.9975\\
0.0011  \qquad  0.0013  \qquad  0.0001  \qquad  0.9975\\
0.0051  \qquad  0.0275  \qquad  0.0005  \qquad  0.9668\\

Wie zu erwarten wird die Zerlegung in die vier Cluster durch das Senken der Zerlegungsunschärfe von 2.0 auf 1.5 schärfer. Große Zugehörigkeitswerte werden noch größer und kleine Zugehörigkeitswerte werden noch kleiner (gehen gegen Null).


\subsection*{d)}

\subsubsection*{2 Cluster}

00.988\qquad	00.012  \\
00.988\qquad	00.012	\\
00.998\qquad	00.002	\\
00.999\qquad	00.001	\\
00.999\qquad	00.001	\\
00.999\qquad	00.001	\\
00.734\qquad	00.266	\\
00.729\qquad	00.271	\\
00.995\qquad	00.005	\\
00.960\qquad	00.040	\\
00.960\qquad	00.040	\\
00.998\qquad	00.002	\\
00.003\qquad	00.997	\\
00.003\qquad	00.997	\\
00.000\qquad	01.000	\\
00.000\qquad	01.000	\\
00.043\qquad	00.957	\\
00.058\qquad	00.942	\\

Partition coefficient:	00.932\\
Entropy:		00.120\\
Non-fuzziness index:	00.865

\subsubsection*{3 Cluster}

00.997\qquad	00.001\qquad	00.003	\\
00.997\qquad	00.001\qquad	00.003	\\
00.990\qquad	00.001\qquad	00.009	\\
00.998\qquad	00.000\qquad	00.001	\\
00.998\qquad	00.000\qquad	00.001	\\
00.997\qquad	00.000\qquad	00.003	\\
00.806\qquad	00.132\qquad	00.062	\\
00.796\qquad	00.138\qquad	00.065	\\
00.003\qquad	00.995\qquad	00.001	\\
00.003\qquad	00.995\qquad	00.001	\\
00.000\qquad	01.000\qquad	00.000	\\
00.000\qquad	01.000\qquad	00.000	\\
00.041\qquad	00.916\qquad	00.043	\\
00.051\qquad	00.889\qquad	00.060	\\
00.002\qquad	00.000\qquad	00.998	\\
00.002\qquad	00.000\qquad	00.998	\\
00.002\qquad	00.000\qquad	00.998	\\
00.030\qquad	00.001\qquad	00.969	\\


Partition coefficient:	00.935\\
Entropy:		00.135\\
Non-fuzziness index:	00.903


\subsubsection*{4 Cluster}

00.998\qquad	00.001\qquad	00.000\qquad	00.001  \\
00.998\qquad	00.001\qquad	00.000\qquad	00.001  \\
00.994\qquad	00.002\qquad	00.000\qquad	00.004	\\
01.000\qquad	00.000\qquad	00.000\qquad	00.000	\\
01.000\qquad	00.000\qquad	00.000\qquad	00.000	\\
00.999\qquad	00.000\qquad	00.000\qquad	00.001	\\
00.173\qquad	00.767\qquad	00.031\qquad	00.028	\\
00.173\qquad	00.765\qquad	00.033\qquad	00.029	\\
00.003\qquad	00.976\qquad	00.018\qquad	00.004	\\
00.004\qquad	00.967\qquad	00.023\qquad	00.006	\\
00.000\qquad	00.000\qquad	00.999\qquad	00.000	\\
00.000\qquad	00.000\qquad	00.999\qquad	00.000	\\
00.000\qquad	00.001\qquad	00.999\qquad	00.000	\\
00.000\qquad	00.001\qquad	00.999\qquad	00.000	\\
00.002\qquad	00.001\qquad	00.000\qquad	00.997	\\
00.001\qquad	00.001\qquad	00.000\qquad	00.997	\\
00.001\qquad	00.001\qquad	00.000\qquad	00.997	\\
00.032\qquad	00.005\qquad	00.000\qquad	00.962	\\

Partition coefficient:	00.945\\
Entropy:		00.117\\
Non-fuzziness index:	00.926

\subsubsection*{Vergleich}

Der Zerlegungskoeffizient und der Non-Fuzzyness-Index steigen leicht mit zunehmender Clusterzahl, die Zerlegungsentropie ist für drei Cluster am größten und für vier am niedrigsten. Also hat die Zerlegung in vier Cluster insgesamt in allen drei Qualitätsmerkmalen das beste Ergebnis und kann daher als im Vergleich optimal angesehen werden.

\subsection*{e)}

00.984\qquad	00.007\qquad	00.007\qquad	00.003	\\
00.984\qquad	00.007\qquad	00.007\qquad	00.003	\\
00.555\qquad	00.141\qquad	00.176\qquad	00.129	\\
00.980\qquad	00.008\qquad	00.006\qquad	00.006	\\
00.980\qquad	00.008\qquad	00.006\qquad	00.006	\\
00.734\qquad	00.077\qquad	00.141\qquad	00.047	\\
00.000\qquad	00.999\qquad	00.000\qquad	00.001	\\
00.001\qquad	00.995\qquad	00.001\qquad	00.003	\\
00.226\qquad	00.086\qquad	00.651\qquad	00.037	\\
00.014\qquad	00.009\qquad	00.970\qquad	00.007	\\
00.001\qquad	00.001\qquad	00.997\qquad	00.001	\\
00.001\qquad	00.001\qquad	00.997\qquad	00.001	\\
00.001\qquad	00.001\qquad	00.998\qquad	00.000	\\
00.032\qquad	00.189\qquad	00.734\qquad	00.045	\\
00.001\qquad	00.007\qquad	00.001\qquad	00.991	\\
00.003\qquad	00.026\qquad	00.003\qquad	00.968	\\
00.015\qquad	00.146\qquad	00.013\qquad	00.825	\\
00.049\qquad	00.070\qquad	00.050\qquad	00.830  \\

Partition coefficient:	00.839\\
Entropy:		00.327\\
Non-fuzziness index:	00.785

\subsubsection*{Vergleich zu d)}

Schon an den Zugehörigkeitswerten kann schnell erkannt werden, dass die Zerlegung mit der "`diagonal norm"' deutlich weniger scharf ist. Dies spiegeln auch die Zerlegungsqualitätsmerkmale wider, der Zerlegungskoeffizient und Non-Fuzzyness-Index sind ca. 11\ % bzw. 15\% niedriger, die Entropie ist fast dreimal so hoch.
